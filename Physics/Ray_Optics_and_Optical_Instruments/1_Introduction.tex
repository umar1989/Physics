\documentclass[11pt,a4paper]{extarticle}

% ===================== Page Setup =====================
\usepackage[margin=1in]{geometry}
\usepackage{fancyhdr}
\usepackage{lastpage}
\usepackage{graphicx}
\usepackage{tikz}
\usepackage{titlesec}
\usepackage{multicol}
\usepackage{enumitem}
\usepackage{hyperref}
\usepackage{caption}
\usepackage{float}

% ===================== Math & Physics =====================
\usepackage{amsmath, amssymb}
\usepackage{physics}
\usepackage{siunitx}
\usepackage{circuitikz}
\usepackage{pgfplots}
\pgfplotsset{compat=1.18}

% ===================== Color and Boxes =====================
\usepackage[most]{tcolorbox}
\tcbuselibrary{listingsutf8}

% Define custom boxes
\newtcolorbox{formulaBox}[1][]{colback=blue!5!white,colframe=blue!75!black,
  title=Important Formula,#1}

\newtcolorbox{tipBox}[1][]{colback=yellow!10!white,colframe=orange!90!black,
  title=Concept Tip,#1}

\newtcolorbox{noteBox}[1][]{colback=green!5!white,colframe=green!50!black,
  title=Note,#1}

% ===================== Header and Footer =====================
\pagestyle{fancy}
\fancyhf{}
\lhead{Physics Notes}
\rhead{CBSE/JEE/NEET}
\rfoot{Page \thepage\ of \pageref{LastPage}}

% ===================== Section Format =====================
\titleformat{\section}{\Large\bfseries}{\thesection}{1em}{}
\titleformat{\subsection}{\large\bfseries}{\thesubsection}{1em}{}

% ===================== Question Format Shortcuts =====================
\newcommand{\question}[2]{\item #1 (#2 mark)\\\textbf{Answer:} #2}
\newcommand{\mcqquestion}[5]{%
  \item #1 \hfill \textbf{}%
  \begin{tasks}(2)
    \task #2
    \task #3
    \task #4
    \task #5
  \end{tasks}
}
\usepackage{tasks}
\setlength{\parindent}{0pt}
\setlength{\parskip}{0.5em}

% ===================== Title =====================
\title{\Huge Physics Master Notes \& Question Paper Template}
\author{Your Name \\ \small Department of Physics \\ \small CBSE / JEE / NEET}
\date{\today}

\begin{document}
\maketitle
\hrule
\vspace{1em}

% ===================== Example Content =====================

\section{Reflection of Light}
\subsection{Laws of Reflection}
\begin{noteBox}
The angle of incidence is equal to the angle of reflection.
\end{noteBox}

\begin{formulaBox}
\[
\theta_i = \theta_r
\]
\end{formulaBox}
\section{Law of Reflection}

\subsection{Ray Diagram}
\begin{center}
\begin{tikzpicture}
  \draw[thick] (-3,0) -- (3,0); % mirror
  \draw[->, thick] (-2,2) -- (0,0) node[below] {O}; % incident ray
  \draw[->, thick] (0,0) -- (2,2); % reflected ray
  \draw[dashed] (0,-1) -- (0,3); % normal
\end{tikzpicture}
\end{center}

\begin{tipBox}
Use ray diagrams to trace paths of light in spherical mirrors.
\end{tipBox}

% ===================== Question Examples =====================
\section*{Sample Questions}

\begin{enumerate}
  \item What is the principle behind reflection of light? (2)
  \textbf{Answer: Reflection follows the law that angle of incidence equals angle of reflection.}

  \item Derive the mirror equation using geometry. (5)\\
  \textbf{Answer: Let us consider a concave mirror... (continue derivation).}

  \item \textbf{Assertion:} The focal length of a concave mirror is negative. \\
        \textbf{Reason:} The focus lies behind the mirror.

  \mcqquestion{The image formed by a convex mirror is always}{Real}{Inverted}{Virtual and Erect}{None}

\end{enumerate}

% ===================== End =====================
\end{document}
